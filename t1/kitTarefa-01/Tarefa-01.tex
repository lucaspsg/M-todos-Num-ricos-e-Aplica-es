\documentclass{article}
\usepackage[utf8]{inputenc}
\usepackage{amsmath}
\usepackage[dvipdfmx]{graphicx} 
\usepackage{bmpsize}
\usepackage{bbm}



\title{MAP3122 - Métodos Numéricos e Aplicações\\ Professor Roma\\ }
\author{TAREFA \#01 --- SEMANA 01\\ Suba até 2023.01.23, 23:55h.}
\date{}

\begin{document}
\maketitle

%\noindent Professor Alexandre Roma\\
%roma ``at'' ime.usp.br\\
%SALA 288-A, telefones 3091-6144 ou 3091-6136 (SecMAP)\\
%Atendimento 3as e 5as feiras - ligue ou passe um e-mail antes.

    \vspace{-3mm}
    
    \mbox{}\hfill  --- VERSÃO 2023.01.03

\vspace{3mm}

Por favor, leia atentamente. Os objetivos principais das tarefas são o aprendizado contínuo e o amadurecimento técnico.
Embora todas as questões  possam ser resolvidas de forma individual, esta tarefa foi preparada para ser resolvida em dupla podendo até mesmo ser discutida coletivamente. Valem as seguintes orientações para a TAREFA \#01:

\begin{enumerate}
 
 \item Todos os alunos devem entregar a solução de todas as questões, independentemente se elas forem resolvidas individualmente, em dupla ou coletivamente.  
 \item Nunca digite as soluções de questões teóricas. Estas devem ser feitas à mão pelo aluno e carregadas no  formato pdf. IMPORTANTE: tenha sempre o original com você para uso em aula (material de consulta).
 \item Questões computacionais, além de um arquivo pdf contendo um relato técnico sucinto (exemplo na página da disciplina), devem ser acom\-panha\-das pelo programa usado. Qualquer linguagem de programação pode ser usada.
 \item Salvo menção em contrário, todo o método numérico usado deve ser programado pela dupla.
 \item Ajude sim, sempre que possível, um colega com dificuldade mas  não preju\-dique o aprendizado dele fornecendo soluções prontas...
 \item Em caso de dúvida, use o Fórum do Estudante. Por favor, não envie suas dúvidas diretamente ao professor.
 Que tenhamos todos um semestre produtivo!
\end{enumerate}
  
\noindent {\bf ATENÇÃO:}\, alguma(s) dentre as questões abaixo pode(m) valer nota e/ou ser(em) incluída(s) em outras atividades durante o semestre.
\mbox{}
\vspace{5mm}

\section{QUESTÃO COMPUTACIONAL}
{\it ``Boa conduta'' no traçado de gráficos.}

\vspace{5mm}

Usando a linguagem com a qual se pretende programar regularmente na disciplina, trace num mesmo sistema cartesiano de eixos  os gráficos de 
\[y(t)=\cos(m\,t),\quad -\pi\le t\le +\pi,\]
para os valores de $m=1,2$ e 3. Convencionaremos como sendo uma ``boa conduta gráfica'' o seguinte:
\begin{enumerate}
 \item Não use cores, pois elas dificultam a impressão em papel caso ela seja necessária.
 \item Curvas diferentes usam linhas com estilos diferentes: ponto-ponto-ponto, traço-ponto-traço, etc. No contexto de equações diferenciais, apenas a solução exata poderá ser representada por linha contínua. 
 \item Coloque legenda para identificar cada curva.
 \item Coloque nomes nos sistemas de eixo e unidades métricas (ou ``adimensional'' se for o caso).
 \item Gráficos têm título!
\end{enumerate}

\section{QUESTÃO COMPUTACIONAL}
{\it Tabelas de convergência numérica para verificar a correta implementação (e para observar o comportamento do método em aplicações para as quais não se tem solução exata conhecida).}

\vspace{5mm}

Ao implementar métodos numéricos para resolver   uma equação diferencial ordinária com condição inicial (Problema de Cauchy), 
\begin{eqnarray}\label{convEuler}
    \begin{array}{rcl}
    \dot{y}(t)  = f(t,y(t)), & & t_0<t\le T,\qquad y(t_0) = y_0,\\
    \end{array} 
\end{eqnarray}
precisamos confeccionar {\it tabelas de convergência numérica} tanto para veri\-ficar a correta implementação do método quanto para observar, posteriormente, o comportamento que o método apresenta no problema de interesse.
A Tabela~\ref{tabmodel-1} é um exemplo típico de um estudo de convergência para verificar a correta implementação de um método de primeira ordem. Para confeccioná-la, usamos um problema com solução exata conhecida: é a estratégia de ``depuração via solução manufaturada''. Por simplicidade, aqui construímos a sequência $\eta(T,h_n)$ de aproximações numéricas da solução exata em $t=T$, $y(T)$, calculadas pelo método implementado  à medida que usa\-mos passos de integração progressivamente menores, $h_n$. A análise é baseada  observando-se como  vai a zero o valor absoluto do {\it erro de discretização global}, 
$\left|e(T, h_n)\right|=\left|y(T)-\eta(T,h_n)\right|,$ reportado na terceira coluna da Tabela~\ref{tabmodel-1}.

\begin{table}[!ht]
\centering
\begin{tabular}{rccc}\hline\hline\\
 $n$ &$h_n=\displaystyle \frac{(T-t_0)}{n}$ & $\left|e(T, h_n)\right|$  & ordem $p$\\\\
	\hline\hline
	\\
  128 &  1.95312e-03 &  1.30556e-11 &  ----------- \\
  256 &  9.76562e-04 &  1.01275e-11 &  3.66390e-01 \\
  512 &  4.88281e-04 &  6.49856e-12 &  6.40090e-01 \\
 1024 &  2.44141e-04 &  3.70427e-12 &  8.10930e-01 \\
 2048 &  1.22070e-04 &  1.98031e-12 &  9.03467e-01 \\
 4096 &  6.10352e-05 &  1.02417e-12 &  9.51272e-01 \\
 8192 &  3.05176e-05 &  5.20845e-13 &  9.75525e-01 \\
16384 &  1.52588e-05 &  2.62646e-13 &  9.87736e-01 \\
		\hline\hline
\end{tabular}
\caption{Método de Euler aplicado ao Problema de Cauchy~(\ref{convEuler}) em $t=T$. 
}
\label{tabmodel-1}
\end{table}

Como veremos, para tais métodos numéricos a {\it ordem de convergência}\, é O$(h^p)$, onde $p$ pode ser estimado por
\[ p\approx  
\frac{ \log\left(\left|e(T,h_{n})/e(T,h_{n+1})\right|\right) }
     { \log\left(h_{n}/h_{n+1} \right)}
,\quad p\in \{0,1,2,..\}.
\]
Em geral, mas não necessariamente sempre, a razão $h_{n}/h_{n+1}=r$, $n=1,2,\ldots$, com $r>1$ constante. Neste caso, temos
\[ p\approx  
\log_r\left(\left|e(T,h_n)/e(T,h_{n+1})\right|\right),\quad p\in \{0,1,2,..\}.
\]

Observando a quarta coluna, vemos que o método implementado se comporta {\it assin\-toticamente}\, como um método de {\it primeira ordem}, isto é, $p$ tende a $1$ à medida que $h$ tende a $0$. Também dizemos que ele é um método de 
{\it ordem 1} (no instante $t=T$). Veja que nesse caso o erro é diretamente propocional a $h$: se o passo de integração cai por 2 o erro também deve cair aproximadamente por 2. Outros detalhes sobre as informações contidas em cada coluna, no momento, não são relevantes pois eles serão vistos {\it exaustivamente} em breve. O objetivo desta questão é preparar o aluno para confeccionar tais tabelas em \LaTeX. 

Para tal, podemos por exemplo formatar a saída de nosso programa e direcioná-la à tela (ou a um arquivo de texto) e depois usar recursos de edição do tipo {\it cut/paste}. O comando \LaTeX\, que gerou a Tabela~\ref{tabmodel-1} encontra-se no arquivo fonte desta tarefa, ``Tarefa-01.tex''. O programa  pode gerar um arquivo de saída do tipo texto contendo
\begin{verbatim}
          128 &  1.95312e-03 &  1.30556e-11 &  ----------- \\
          256 &  9.76562e-04 &  1.01275e-11 &  3.66390e-01 \\
              .
              .
              .    
        16384 &  1.52588e-05 &  2.62646e-13 &  9.87736e-01 \\
\end{verbatim}
que, posteriormente, podemos recortar e colar no arquivo .tex o qual já estaria preparado com os comandos
\begin{verbatim}
\begin{table}[!ht]
    \centering
    \begin{tabular}{rccc}
    \hline\hline\\
          $n$ & $h_n=\displaystyle \frac{(T-t_0)}{n}$ 
              & $\left|e(T, h_n)\right|$  & ordem $p$\\\\
   \hline\hline\\
   
   [RECORTE E COLE SEUS RESULTADOS AQUI]
   
   \hline\hline \end{tabular}
   \caption{SUA LEGENDA AQUI.}\label{SEU LABEL AQUI}
\end{table}
\end{verbatim}

Não é obrigatório, mas você pode partir dos programas e dos fontes \LaTeX\quad   disponibilizados na página da disciplina (material suplementar) e no ``kitTarefa-01.zip'' para resolver essa questão. Faça, no mínimo, o seguinte:

\begin{enumerate}
 \item Manufature um problema de valor inicial com apenas uma variável de estado que tenha solução ``suave'' conhecida, confeccione a tabela de convergência usando o erro de discretização global no instante de tempo final e observe a ordem exibida pelo Método de Euler à medida que o passo de integração vai a zero.
 \item Repita o procedimento para um problema que tenha duas variáveis de estado (bidimensional), diferente do sistema massa-mola. Use a norma do máximo para o estudo de convergência.
 \item Use o Método de Euler para resolver um problema bidimensional (com duas variáveis de estado) por exemplo o de presa-predador (equação de Lotka-Volterra) e observe o comportamento de convergência exibido pelo método à medida que o passo de integração vai a zero. Use a norma euclididana para construir sua tabela.
\end{enumerate}

\vspace{3mm}
Inclua traçados das variáveis de estado para vários passos de integração, sempre seguindo a ``boa conduta gráfica''. Faça comentários e análises pertinentes.

\vspace{3mm}\noindent
{\bf IMPORTANTE: é preciso discutir mais e detalhar, é claro. Tais discussões serão feitas sob demanda dos alunos. Os resultados devem ser reportados de forma organizada e de maneira que possam ser reproduzidos por qualquer colega da turma. Há um exemplo de como reportar na página da disciplina (``material suplementar'') que pode servir de ponto de partida caso queira. É esperado apenas um relato curto.}

\section{COMENTÁRIOS FINAIS}
    Acho importante dar um caráter ``instrumental'' à disciplina. O uso do \LaTeX\quad (e.g. via overleaf.com) e a ``boa conduta'' no traçado de gráficos têm este propósito. Boas apresentações valorizam (e muito) a produção intelectual e o esforço de vocês. Trabalho coletivo saudável melhora a famosa ``dinâmica de grupo''.
\end{document}


